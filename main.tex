\documentclass[a4paper,man,natbib]{apa6}

\usepackage[spanish]{babel}
\usepackage[utf8x]{inputenc}
\usepackage{amsmath}
\usepackage{graphicx}
\usepackage[colorinlistoftodos]{todonotes}

\usepackage[hyphens]{url}
\usepackage[hidelinks]{hyperref}
\hypersetup{breaklinks=true}
\urlstyle{same}

\shorttitle{Tarea 6 - Problemas y conflictos ambientales}

\begin{document}

\begin{titlepage}
    \begin{center}
        \vspace*{1cm}

        \begin{Huge}
            \shorttitle{Tarea 6 - Problemas y conflictos ambientales}
        \end{Huge}

        \includegraphics[width=0.22\textwidth]{img/universidadDelValle.png}
        
        \vfill
        \textbf{Tarea 6 - Problemas y conflictos ambientales}\\
        \vfill
        
        Estudiantes\\
        Diego Fernando Chaverra Castillo 201940322\\
        Juan Camilo Santa Gomez 201943214\\
        Sebastián Afanador Fontal 201629587\\
        \vfill
        Profesora\\
        Diana Marcela Mendoza. Mgs
        
        
        \vfill

        \textbf{Introducción a la Gestión Ambiental}
        
        \vfill
           
        Facultad de Ingeniería\\
        Escuela de Ingeniería de Recursos Naturales y del Ambiente\\
        Universidad del Valle\\
        Cali - Colombia\\
        \vfill
        08 de Julio del 2022

    \end{center}
\end{titlepage}

\section{Introducción}
El contenido de este documento es un ejecicio académico. El objetivo es hacer una corta reflexión de un problema y un conflicto ambiental proponiendo una posible solución en cada caso. A continuación el problema y conflicto ambiental respectivamente.

\section{Problema ambiental a nivel local}
\subsection{Residuos sólidos}
Disponer incorrectamente los residuos sólidos es una labor colectiva qué aparentemente no genera un problema grave. Vivimos como si no hubiera un mañana, y como si lo que producimos no causara algún efecto en nuestro entorno. Nos hemos acostumbrado tanto a esta situación que no vemos esto como un problema sino como algo normal, no nos importan los malos olores, que nuestras cañerías se dañen y si no nos preocupa esto menos nos interesa lo que dichas situaciones causan al medio ambiente. Por la falta de educación en este aspecto nos estamos matando a nosotros mismos sin siquiera darnos cuenta pero lo importante no es quejarnos de la situación, sino dar soluciones.\newline

El estado tiene la responsabilidad en la educación y legislación qué regula la disposición de residuos sólidos. La mayoría de los problemas en la recolección transporte y disposición de residuos sólidos se debe a que nuestra región no cuenta con la infraestructura y los procedimientos correctos para dar la solución definitiva. Algunas soluciones a este problema ambiental están relacionadas con crear una ley que penaliza con multas a las personas qué no separen los residuos en sus respectivos contenedores, también crear una flota de camiones especializados por cada tipo de contenedor donde a cada camión solo le corresponde transportar y disponer un tipo de residuo sólido. Toda esta corrección no surtirá frutos a largo plazo si no se enseña a los niños desde los colegios el proceso de separación de los residuos.\newline

\section{Conflicto ambiental a nivel global}
\subsection{Cultivos ilícitos en Colombia}

La tierra es un activo que naturalmente se usa para producir alimentos, sin embargo en Colombia es más rentable usar la tierra para cultivos ilícitos. Colombia siendo un país pequeño a nivel mundial es conocido no por sus hermosas montañas, sus paisajes o su biodiversidad, sino porque es según la ONU uno de los primeros paises productores de cocaína en el mundo \citep{Informed31:online}. Esta realidad, siempre ha sido un conflicto latente en el país que ahora ha escalado a ser un conflicto mundial. Pensando un poco en las posibles soluciones a este conflicto existe la posibilidad de combatirlo mejorando la tecnología en la agricultura; El momento en qué un bulto de café,  papa, maíz o quizás la producción de cualquier otro producto distinto de la coca o la marihuana sea más rentable para un campesino quizás solo en ese momento no tenga tanto sentido seguir sembrando cultivos ilícitos.\newline

Legalizar y normalizar el mercado de los narcóticos es un paso grande para la erradicación de los cultivos ilícitos. Pensar en legalizar la producción y distribución de narcóticos podría ser una medida qué logre reducir la actividad ilícita y la remuneración económica qué tendrían los campesinos al producir este tipo de cultivos, la razón tiene su raíz en la competencia qué se podría generar en este mercado, en contra de esta decisión legislativa seguramente se encuentran los intereses de muchos individuos con poder económico y una cultura qué penaliza el consumo de narcóticos para evitar la pérdida de la cordura. Las causas y los efectos no están sucedidos en el tiempo y el espacio, la solución de hoy podría ser el problema de mañana.\newline


\Urlmuskip=0mu plus 1mu\relax
\bibliographystyle{apalike}
\bibliography{references}


\end{document}

%
% Please see the package documentation for more information
% on the APA6 document class:
%
% http://www.ctan.org/pkg/apa6
%